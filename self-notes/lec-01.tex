\documentclass[a4paper]{article}
\usepackage[english]{babel}
\usepackage[a4paper,top=2cm,bottom=2cm,left=2cm,right=2cm,marginparwidth=1.75cm]{geometry}
\usepackage{amsmath}
\usepackage{amsfonts}
% \usepackage{amsthm}
\usepackage{amssymb}
\usepackage{graphicx}
\usepackage[colorinlistoftodos]{todonotes}
\usepackage[colorlinks=true, allcolors=blue]{hyperref}
\usepackage{import}
\usepackage{pdfpages}
\usepackage{transparent}
\usepackage{xcolor}
\usepackage{algorithmicx}
\usepackage{algpseudocode}

\usepackage{thmtools}
\usepackage{enumitem}
\usepackage[framemethod=TikZ]{mdframed}

\usepackage{xpatch}

\usepackage{boites}
\makeatletter
\xpatchcmd{\endmdframed}
{\aftergroup\endmdf@trivlist\color@endgroup}
{\endmdf@trivlist\color@endgroup\@doendpe}
{}{}
\makeatother

%\usepackage[poster]{tcolorbox}
%\allowdisplaybreaks
%\sloppy

\usepackage[many]{tcolorbox}

\xpatchcmd{\proof}{\itshape}{\bfseries\itshape}{}{}

% to set box separation
\setlength{\fboxsep}{0.8em}
\def\breakboxskip{7pt}
\def\breakboxparindent{0em}

\newenvironment{proof}{\begin{breakbox}\textit{Proof.}}{\hfill$\square$\end{breakbox}}
\newenvironment{ans}{\begin{breakbox}\textit{Answer.}}{\end{breakbox}}
\newenvironment{soln}{\begin{breakbox}\textit{Solution.}}{\end{breakbox}}

% \tcolorboxenvironment{proof}{
%     blanker,
%     before skip=\topsep,
%     after skip=\topsep,
%     borderline={0.4pt}{0.4pt}{black},
%     breakable,
%     left=12pt,
%     right=12pt,
%     top=12pt,
%     bottom=12pt,
% }
%
% \tcolorboxenvironment{ans}{
%     blanker,
%     before skip=\topsep,
%     after skip=\topsep,
%     borderline={0.4pt}{0.4pt}{black},
%     breakable,
%     left=12pt,
%     right=12pt,
% }

\mdfdefinestyle{enclosed}{
    linecolor=black
    ,backgroundcolor=none
    ,apptotikzsetting={\tikzset{mdfbackground/.append style={fill=gray!100,fill opacity=.3}}}
    ,frametitlefont=\sffamily\bfseries\color{black}
    ,splittopskip=.5cm
    ,frametitlebelowskip=.0cm
    ,topline=true
    ,bottomline=true
    ,rightline=true
    ,leftline=true
    ,leftmargin=0.01cm
    ,linewidth=0.02cm
    ,skipabove=0.01cm
    ,innerbottommargin=0.1cm
    ,skipbelow=0.1cm
}

\mdfsetup{%
    middlelinecolor=black,
    middlelinewidth=1pt,
roundcorner=4pt}

\setlength{\parindent}{0pt}

\mdtheorem[style=enclosed]{theorem}{Theorem}
\mdtheorem[style=enclosed]{lemma}{Lemma}[theorem]
\mdtheorem[style=enclosed]{claim}{Claim}[theorem]
\mdtheorem[style=enclosed]{ques}{Question}
\mdtheorem[style=enclosed]{defn}{Definition}
\mdtheorem[style=enclosed]{notn}{Notation}
\mdtheorem[style=enclosed]{obs}{Observation}
\mdtheorem[style=enclosed]{eg}{Example}
\mdtheorem[style=enclosed]{cor}{Corollary}
\mdtheorem[style=enclosed]{note}{Note}

% \let\thetheorem=\relax
% \let\thelemma=\relax
% \let\theclaim=\relax
% \let\theques=\relax
% \let\thedefn=\relax
% \let\thenotn=\relax
% \let\theobs=\relax
% \let\thecor=\relax
% \let\thenote=\relax

% \renewcommand\qedsymbol{$\blacksquare$}
\newcommand{\nl}{\vspace{0.2cm}\\}
\renewcommand{\sp}{\vspace{0.2cm}}
\newcommand{\ol}{\overline}
\newcommand{\mc}{\mathcal}
\newcommand{\mi}{\mathit}
\newcommand{\mf}{\mathbf}
\newcommand{\mb}{\mathbb}
\renewcommand{\L}{\mc{L}}
\newcommand{\F}{\mathcal{F}}
\renewcommand{\P}{\mathcal{P}}
\newcommand{\N}{\mathcal{N}}
\newcommand{\R}{\mathbb{R}}
\newcommand{\changesto}{\vdash}
\newcommand\Vtextvisiblespace[1][.3em]{%
    \mbox{\kern.06em\vrule height.3ex}%
    \vbox{\hrule width#1}%
    \hbox{\vrule height.3ex}
}
\newcommand{\blank}{{\Vtextvisiblespace[0.7em]}}
\newcommand{\leftend}{\triangleright}
\newcommand{\comp}{\overline}


\newcommand{\incfig}[1]{%
    \def\svgwidth{\columnwidth}
    \import{./figures/}{#1.pdf_tex}
}
\pdfsuppresswarningpagegroup=1

\title{\textbf{Stochastic of Finance Lecture 1}}
\date{}

%\section{Recap}
%
%\section{Definitions}
%
%\begin{defn}
%\end{defn}
%
%\section{Content}
%
%\begin{theorem}
%\end{theorem}
%\begin{proof}
%\end{proof}
%
%\begin{ques}
%\end{ques}
%
%\begin{eg}
%\end{eg}
%
%\begin{claim}
%\end{claim}

\begin{document}
\maketitle
\tableofcontents

\section{Recap}
\section{Content}

\begin{defn}
    \textbf{Axiomatic definition of probability spaces}\nl
    Let $\Omega$ be the set of all possible outcomes of a random experiment. Let $\F \subseteq 2^\Omega$ be a $\sigma$-algebra on $\Omega$, i.e., the following hold:

    \begin{enumerate}
        \item $\Omega \in \F$.
        \item $A \in \F \implies \Omega \setminus A \in \F$.
        \item $\F$ is closed under countable union, that is, if $A_i \in \F$ for $i = 1, 2, \ldots$, then $\cup_{i=1}^\infty A_i \in \F$.
    \end{enumerate}

    Let $P$ be a function (measure) $P : \F \to [0, 1]$ such that the following hold:

    \begin{enumerate}
        \item $P(A) \ge 0$ (trivially holds due to range).
        \item $P$ is $\sigma$-additive, i.e., if $\{A_i\}_{i=1}^\infty \subseteq \F$ is a countable collection of disjoint subsets, then we have $P(\cup_{i=1}^\infty A_i) = \sum_{i=1}^\infty P(A_i)$.
        \item $P(\Omega) = 1$.
    \end{enumerate}

    Then $(\Omega, F, P)$ is a probability space.
\end{defn}
\sp
The classical probability comes from the formal definition when $\Omega$ is finite, $|\Omega| = n$, $P(\{\omega\}) = \frac{1}{n}$ where $\omega \in \Omega$, $\F$ is the power set of $\Omega$,
which also turns out to be a $\sigma$-field on $\Omega$.\nl

\begin{defn}
    \textbf{Random variables}:\nl
    Given a probability space $(\Omega, \F, P)$, if $X : \Omega \to \R$ is a function such that $X^{-1}((-\infty, x]) \in \F$ $\forall x \in \R$, then $X$ is a random variable, or a
    measurable function w.r.t. $\F$.
\end{defn}

\begin{eg}
    If $F$ is the largest $\sigma$-field (power set but maybe for infinite sets), then any real-function is a random variable.
\end{eg}
\begin{eg}
    A constant function is always a random variable.
\end{eg}
\sp
\begin{defn}
    \textbf{Stochastic processes}:\nl
    A stochastic process $\{X(t), t \in T\}$ is a collection of random variables defined on the probability space $(\Omega, \F, P)$.
\end{defn}
\sp
\begin{eg}
    \textbf{Examples of real-life stochastic processes}:
    \begin{enumerate}
    \item Price of some stock at the end of the day.
    \item Number of trades made every second.
    \item Market index at time $t$.
    \item Number of companies registered in stock market at the end of the week.
    \item Variance in a stock price in a day - measure on the random variables (since computed from the data). Usually we call observed information (and not computed information) a random
        variable. Note also time series. Nothing wrong with calling this a random variable, but this won't be the focus of the course.
    \end{enumerate}
\end{eg}
\sp
Some stochastic processes have some important properties, as follows:
\begin{enumerate}
    \item Independence (mutual, not pairwise) - can verify such assumptions.
    \item Stationary - many times we can assume that data is stationary. Two types:
        \begin{enumerate}
            \item Wide sense
            \item Strict sense (by default)
        \end{enumerate}
    \item Memoryless property.
    \item Martingale property - also useful with conditional expectations.
\end{enumerate}
For more, revisit MTL106.
Time homogeneous is similar to stationary.
\sp
\begin{eg}
    \textbf{Poisson process}\nl
    $\{N(t), t \ge 0\}$ - number of events occuring upto and including time $t$. Suppose $N(t) \sim \P(\lambda t)$ where $\P$ is the Poisson distribution, and $\lambda$ is a fixed
    parameter. This stochastic process is called a Poisson process. Some properties:
    \begin{enumerate}
        \item Increments are independent.
        \item Increments are stationary.
        \item Satisfies the memoryless property.
        \item Doesn't satisfy the martingale property.
    \end{enumerate}
    We can derive a random variable that satisfies the martingale property from any random variable.
\end{eg}
\sp
\begin{eg}
    \textbf{Brownian motion/Wiener process}\nl
    Let $\{W(t), t \ge 0\}$ be a stochastic process which satisfies the following conditions:
    \begin{enumerate}
        \item $W(0) = 0$
        \item For fixed $t$, $W(t) \sim \N(0, t)$
        \item Increments are independent.
        \item Increments are stationary.
    \end{enumerate}
\end{eg}

\end{document}
